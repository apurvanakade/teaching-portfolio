\subsection*{Teaching Philosophy}
  While researching my IBL classes, I came across a quote which has since become my guiding principle - ``the goal of teaching is learning, not teaching''.
  I use active learning techniques in all my classes. I believe that the best way to learn math is through making mistakes, getting confused, and struggling toward a solution. I consider myself a coach and a facilitator and teach with the core philosophy that my primary goal is to provide my students with a welcoming and inclusive environment where experimentation is encouraged and honest mistakes aren’t penalized. I believe a math classroom is a place for students, and even the instructor, to grow as mathematicians.

  My teaching experience ranges from creating advanced electives and short bootcamp courses for small groups of students, to managing and teaching in-person courses with several hundred students, to adapting large service courses for asynchronous education.  I have taught topics spanning calculus, linear algebra, discrete math, proof checking using computers, manifolds, and topology.

\subsection*{UWO}
  My most challenging teaching project has been the fully asynchronous Discrete Math course that I'm currently teaching, which has nearly 200 students. 
  I've had to redevelop every aspect of the course from scratch to fit the current unprecedented situation.

  Because this is an overwhelming time for many students, my priority has been to alleviate student anxiety. 
  I've spread out the grades more evenly throughout all assessments and have informed my students beforehand that the exams are going to test competence and not expertise.
  I maintain an active discussion forum on Piazza to develop a sense of an online community.
  Following design principles, I've made my course as easy to follow as possible and all assignments, forums, exams, deadlines, textbook, video lectures, and course resources are available in one place.
  Over the summer I explored several textbooks and chose an online one that allowed me to provide an active learning environment asynchronously.
  Each week, I make short introductory videos and embed them directly at the start of each section along with learning objectives and any special comments.
  I assign weekly, short auto-graded assignments within the textbook which provide immediate feedback.
  I maintain a blog (\url{https://apurvanakade.github.io/etc-.html}) to document and reflect upon my teaching.

  For the exams, my focus has been on making sure that students who do not cheat are not at a disadvantage while still ensuring that students' mental health is not adversely affected by the pressures on online assessments.
  In a team consisting of I, my postdoc advisor, and two students, we created a randomized pool of hundreds of questions on WebWork (open source platform supported by MAA).

  In the previous year, for the first time, I taught in huge coordinated courses where I was the sole point of contact for over 150 students.
  % This was a challenging and rewarding learning experience and drastically changed my approach toward teaching.
  This challenging but rewarding learning experience taught me how to be more inclusive and considerate of students from very diverse backgrounds and to ensure that even those who had other priorities could benefit from learning math in whatever way possible.
  I made myself available by having a lot of office hours, which often ended up having high attendance.
  I memorized as many of my students' names as I could to connect with them on a personal level;
  I learned this skill while teaching at a 5 week summer program, Mathcamp, where I learned to memorize the names of 120 students in an extremely short time span.
 
\subsection*{Mentoring}
  At JHU, I participated in a Directed Reading Program that pairs undergraduate students with graduate students/junior faculty to undertake independent study projects as a mentor and a co-organizer. 
  After moving to UWO, I started a DRP chapter here with the help of one of my colleagues. 
  Our motivation has slowly morphed from trying to get more students to major in math to build a thriving and welcoming undergraduate math community.
  These programs have been successful at both the universities and we've seen significant and continuing growth in student interest and participation.
  
  \newpage
  This semester I'm also organizing a small topology bootcamp which is a crash course on point-set topology for undergraduates who'll be taking the graduate Algebraic Topology course that I'll be teaching next semester. I am also helping out the graduate students at UWO start a math circle styled outreach program for the local schools. This has been quite challenging given the current circumstances but we have been making slow but steady progress. Adhering to my principle of letting students figure things out for themselves, I'm participating in an advisory capacity and providing feedback and direction and trusting them to do the decision-making.

  
\subsection*{JHU}

  At JHU, I was fortunate to be given an opportunity to develop and teach Honors Single Variable Calculus for two semesters. 
  The class size went up from 4 in my first year, to 10 in my second, and then close to 20 in the year after. 
  I also developed and taught two intersession courses introducing non-math majors to some algebraic topology and Galois theory. 
  While preparing the materials from scratch, I learned a lot about looking at things from a student's perspective.
  All these classes were structured in an inquiry-based format and my goal was to get students to understand the messy process of discovery in math rather than just provide them with the misleading sanitized version of it.
  % Looking back, I should have used the available resources online to better 
  % Also write about intersession classes.
  
  
  
  
  
  
  

\subsection*{USA/Canada Mathcamp}
  My biggest influences have come from being a mentor (2017-20) and an academic coordinator
  (2018) at the Canada/USA Mathcamp, a summer program for high school students.
  Mathcamp gave me an opportunity to be a part of a loving and caring community, to be surrounded by people who love math and love to teach it and excel at it.
  I took on the role of an academic coordinator to contribute back to Mathcamp, to challenge myself, and to learn more about teaching. 
  % Some selected topics that I taught are: Manifolds, Euler Characteristic, Gaussian Curvature, Cech cohomology, Lie groups, Galois Theory.
  The academic coordinators are responsible for designing and running all the academic activities, including, inviting and hosting external visitors, designing a balanced five-week class schedule (nearly 60 classes), assigning (110) students to projects, and teaching. 

  I am very much aware of the biases unconsciously ingrained in the culture of a math classroom which discourages women and minorities. 
  I became especially cognizant of this at Mathcamp where we have serious conversations about inclusiveness, especially as a part of the staff hiring committee.
  In my classroom, I try to mitigate these biases by deemphasizing the role of genius and discouraging aggressive competitiveness, instead encouraging persistence and hard work. 
  I make sure that everyone gets a chance to speak in class and participate and collaborate in the communal experience of doing math.

\subsection*{Professional Development}
  I am a member of the Project NExT'20 cohort, a professional development program sponsored by the MAA for math educators at the university level.
  Participating in the summer workshop exposed me to a wonderful community of math educators from whom I hope to learn from in the long run.
  I've already found fellow Discrete Math educators with whom I share teaching tips on the MAA Connect forum.
  I have completed a certification course at the Teaching Academy at JHU where I learned about several important pedagogical concepts such as inquiry-based learning, backward course design, learning objectives, etc. which I regularly incorporate into my own teaching.
  I also attend the workshops and seminars organized at the Center for Teaching and Learning at UWO. 

\subsection*{Personal experiences}
  I consider myself an avid learner and keen listener and observer. 
  My current course design is influenced by the online courses I've taken on Coursera and Kaggle.
  I was introduced to active learning methods through a brilliant French language class I took during my grad school.
  I am a regular practitioner of yoga and (pre-Covid) improv comedy and various forms of dance. 
  These non-academic interests help me take a holistic approach toward my pedagogy.

  My next big goal is to figure out ways to quantitatively assess my teaching and pedagogical effectiveness and concretely implement evidence-driven methodologies in my classrooms.
  I also interested in understanding how scale effects pedagogy and how insights from psychological research can be used to enhance student learning.